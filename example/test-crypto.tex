\documentclass[11pt]{article}
% Utiliser le style trustnocorpo embarqué
\usepackage{trustnocorpo-spacial}

\usepackage[utf8]{inputenc}
\usepackage{geometry}
\usepackage{hyperref}

\title{TrustNoCorpo (tnc) — Démonstration}
\author{TrustNoCorpo}
\date{\today}

\begin{document}
\maketitle

\section{Introduction}

Ce document démontre le système de traçabilité cryptographique de \textbf{TrustNoCorpo (tnc)}. À chaque génération, tnc intègre des métadonnées (hash de build, classification, informations d'origine) directement dans le PDF et journalise l'opération dans une base chiffrée.

\section{Fonctionnalités}

\begin{itemize}
    \item Traçabilité automatique de chaque build (horodatage, utilisateur, machine)
    \item Classification configurable (ex. \texttt{CONFIDENTIAL}, \texttt{SECRET})
    \item Intégration de métadonnées via le style \texttt{trustnocorpo-spacial}
    \item Hash unique par génération pour vérification ultérieure
\end{itemize}

\section{Informations de Sécurité}

Ce document contient des informations de traçabilité intégrées par tnc permettant de :
\begin{enumerate}
    \item Identifier qui a généré le document
    \item Connaître le moment exact de génération
    \item Tracer la machine d'origine
    \item Vérifier l'intégrité via le hash
\end{enumerate}

\section{Décodage}

Les informations de traçabilité et la classification sont intégrées automatiquement via le style \texttt{trustnocorpo-spacial}. Utilisez la CLI \texttt{trustnocorpo} pour vérifier et lister les builds.

\end{document}
